% Typesetting
\usepackage[T1]{fontenc}
\usepackage[utf8]{inputenc}
\usepackage[english]{babel}
%\usepackage{lmodern} % latin modern font
\usepackage{csquotes} % pro­vides ad­vanced fa­cil­i­ties for in­line and dis­play quo­ta­tions (better to load when using biblatex)
\usepackage{textcomp} % pro­vide many text sym­bols (such as baht, bul­let, copy­right, mu­si­cal­note, onequar­ter, sec­tion, and yen), in the TS1 en­cod­ing
%\usepackage{setspace}
%\onehalfspacing % 1.5 linespaceing
%\usepackage{fancyhdr} % pro­vides ex­ten­sive fa­cil­i­ties, both for con­struct­ing head­ers and foot­ers, and for con­trol­ling the

\usepackage{siunitx}
\sisetup{
	group-digits = integer, % only group digits (by three) for integers (not decimals)
	binary-units = true, % load binary units
	detect-all
}
%\usepackage{enumitem}
%\setlist[enumerate]{label*=\arabic*.,topsep=0pt,partopsep=0pt,parsep=0pt,itemsep=0pt}
%\setlist[itemize]{topsep=0pt,partopsep=0pt,parsep=0pt,itemsep=0pt}
%\usepackage{bigfoot} % The pack­age aims to pro­vide a ‘one-stop’ so­lu­tion to re­quire­ments for foot­notes
%\usepackage{afterpage} % to use \footnotemark and \footnotetext in captions for special cases
\usepackage{algorithm} % the al­go­rithm pack­age de­fines a float­ing al­go­rithm en­vi­ron­ment de­signed to work with the al­go­rith­mic style
\usepackage{algorithmic}
%\usepackage{algpseudocode} % The algorithmicx package provides many possibilities to customize the layout of algorithms.
\usepackage{newunicodechar}
\usepackage{pifont}

% Math
\usepackage{amsmath}
\usepackage{amsfonts}
\usepackage{amssymb}
\usepackage{amsthm}
\usepackage{bm}
%\usepackage{mathtools}

% Figures
\usepackage{graphicx}
\graphicspath{{figures/}}
\usepackage{xcolor}
%\usepackage[font=small, labelfont=bf, format=plain, labelsep=space, figurename=Figure, tablename=Table, skip=5pt]{caption}
%\usepackage[labelfont=rm, labelformat=parens, labelsep=space, skip=2pt]{subcaption}
\usepackage[skip=2pt]{subcaption}
%	IEEE caption set-up
\captionsetup[figure]{name=Fig.}
\captionsetup[table]{name=TABLE, font=footnotesize, labelfont=up, textfont=sc, labelsep=newline, justification=centering}
\captionsetup[subfigure]{labelformat=simple}
\renewcommand*{\thesubfigure}{(\alph{subfigure})} % adds parens around subfigure label for clean subreference using \ref{} or \subref{}. Must be used with subcaption options, labelformat=simple and subrefformat=simple (not sure to be required...)
\usepackage{tikz}
%\usepackage{pgfgantt}

% Tables
\usepackage{multirow}
%\usepackage{adjustbox}
%\usepackage{makecell}
\usepackage{longtable} % use of \linebreak instead of \\ in headers to avoid a bug with longtables (or longtabu) across two pages
\usepackage{tabu}
%	\tabulinesep = 4pt
	\tabcolsep = 4pt
\usepackage{booktabs}
%\usepackage{colortbl}

% Others
%\usepackage[draft]{pdfpages} % include pdf pages
\usepackage{calc}
%\usepackage{rotating}
\usepackage{todonotes} % \todo, \missingfigures and \listoftodos
\usepackage{xifthen} % This pack­age ex­tends the ifthen pack­age by im­ple­ment­ing new com­mands to go within the first ar­gu­ment of \ifthenelse
\usepackage{xparse} % The pack­age pro­vides a high-level in­ter­face for pro­duc­ing doc­u­ment-level com­mands. 
\usepackage{etoolbox} % The package is a toolbox of programming facilities geared primarily towards LaTeX class and package authors. 
\usepackage{xstring}
\usepackage{xspace} % useful for newcommands and spacing

% References and URLs
\usepackage{cleveref}
\crefname{table}{Table}{Tables} % default: table
\crefname{figure}{Fig.}{Fig.} % default: figure
\crefname{section}{Section}{Section} % default: section
\crefname{equation}{Equation}{Equation} % default: eq.
\usepackage{url}
